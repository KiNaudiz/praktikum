\documentclass[10pt,a4paper]{scrartcl}
\usepackage[utf8x]{inputenc}
\usepackage[english,german]{babel}
\usepackage{amsmath}
\usepackage{amssymb}
\usepackage{graphicx}
\usepackage{listings}
\usepackage[format=plain,indention=1cm,font=sf ,labelfont=bf, nooneline,center]{caption}
\renewcommand{\captionfont}{\sffamily \slshape}
\renewcommand{\captionlabelfont}{ \sffamily \slshape \bfseries   }
\usepackage{subcaption}
\usepackage{url}
\usepackage{empheq}
\usepackage{dcolumn}
\usepackage{rotating}
\newcolumntype{d}{D{.}{.}{-1} }

\usepackage{tikz}
\usetikzlibrary{circuits.ee.IEC,matrix,
circuits.logic.US,
circuits.logic.IEC,
circuits.logic.CDH}

% Quelle https://trucastuces.wordpress.com/2012/10/10/boxed-equations-in-latex/
\newlength\dlf  % Define a new measure, dlf
\newcommand\alignedbox[2]{
    % Argument #1 = before & if there were no box (lhs)
        % Argument #2 = after & if there were no box (rhs)
        &  % Alignment sign of the line
        {
            \settowidth\dlf{$\displaystyle #1$}
            % The width of \dlf is the width of the lhs, with a displaystyle font
                \addtolength\dlf{\fboxsep+\fboxrule}
            % Add to it the distance to the box, and the width of the line of the box
                \hspace{-\dlf}
            % Move everything dlf units to the left, so that & #1 #2 is aligned under #1 & #2
                \boxed{#1 #2}
            % Put a box around lhs and rhs
        }
}

\title {Physikalisches Fortgeschrittenenpraktikum I\linebreak
Halbleiterbauelemente}
\author {\emph{Clemens Kurzenberg}\\212204196}
\date {30.04.2015}

\begin{document}

\maketitle

\begin{abstract}
    %TODO
\end{abstract}

\tableofcontents

\pagebreak
\listoffigures
\listoftables

\pagebreak
\section {Einführung}

\subsection {Eigenschaften von Operationsverstärkern}

\begin{figure}[!ht]
    \centering
    \begin{tikzpicture}[%show background rectangle,
    circuit ee IEC, circuit symbol lines/.style={draw,thick},
    font=\sffamily\upshape, circuit logic US, %huge circuit symbols,
    >=latex % Voreinstellung für Pfeilspitzen
    ]
        \matrix (S) [
            matrix of nodes, nodes in empty cells,
            inner sep=0pt, outer sep=-.5\pgflinewidth,
            column sep=10mm, row sep = \tikzcircuitssizeunit,
            nodes={minimum width=0pt}
        ]
        {
            &&&&&&  \\
            &&&&&&  \\
            &&&&&&  \\
            &&&&&&  \\
            &&&&&&  \\
            &&&&&&  \\
            &&&&&&  \\
        };

        % Bauteile
        \draw   (S-3-2) to
                (S-1-2) to [resistor={info=$R_g$, name=Rg}](S-1-6)
                (S-3-2) to [resistor={info=$R_T$, name=RT}](S-7-2)
                (S-3-2) to [voltage source={info=$U_{img}$}, name=Ui](S-3-4)
                        ;
        \node [buffer gate={}](OP) at (S-4-5) {} ;

        \draw[shorten >=1\tikzcircuitssizeunit] (S-3-4) --  (S-3-5)node[left=2.5pt]{\footnotesize-};
        \draw[shorten >=1\tikzcircuitssizeunit] (S-5-4) -- (S-5-5)node[left=1.75pt]{\tiny +};
        \draw[shorten <=2\tikzcircuitssizeunit] (S-4-5)% node[right=4.5pt]{\tiny +} --
        -- (S-4-7);

        % Leiter
        \draw   (S-3-1) -- (S-3-2)
                (S-3-1) -- (S-3-4)
                ;

        % Klemmen, Knoten
        \draw [fill=white]  (S-3-1) circle (2pt)
                            % (S-6-1) circle (2pt)
                            (S-4-7) circle (2pt)
                            % (S-3-4) circle (2pt)
                            ;

        % \draw [fill=black]  (S-1-3) circle (2pt)
        %                     (S-3-3) circle (2pt);

        \draw   (S-3-1) node [anchor=east] {$U_e$}
                (S-4-7) node [anchor=west] {$U_a$};
    \end{tikzpicture}
    \caption{Halbleiterdiode im Wechselstromkreis} \label{fig:HL_Diode}
\end{figure}


\section {Durchführung}

\section {Auswertung}



\pagebreak
% \section {Quellen}
% \begin{thebibliography}{999}
% \bibitem {WalterHerms} G. Walter und G. Herms, Einführung in die Behandlung von Messfehlern -- Ein Leitfaden für das Praktikum der Physik, Universität Rostock 2006d
% \end{thebibliography}

\section {Anhang}


\end{document}
%sagemathcloud={"zoom_width":100}
