\documentclass[10pt,a4paper]{scrartcl}
\usepackage[utf8x]{inputenc}
\usepackage[english,german]{babel}
\usepackage{amsmath}
\usepackage{amssymb}
\usepackage{graphicx}
\usepackage{listings}
\usepackage[format=plain,indention=1cm,font=sf ,labelfont=bf, nooneline,center]{caption}
\renewcommand{\captionfont}{\sffamily \slshape}
\renewcommand{\captionlabelfont}{ \sffamily \slshape \bfseries   }
\usepackage{subcaption}
\usepackage{url}
\usepackage{empheq}
\usepackage{dcolumn}
\usepackage{rotating}
\newcolumntype{d}{D{.}{.}{-1} }

\usepackage{tikz}
\usetikzlibrary{circuits.ee.IEC,matrix}

% Quelle https://trucastuces.wordpress.com/2012/10/10/boxed-equations-in-latex/
\newlength\dlf  % Define a new measure, dlf
\newcommand\alignedbox[2]{
    % Argument #1 = before & if there were no box (lhs)
        % Argument #2 = after & if there were no box (rhs)
        &  % Alignment sign of the line
        {
            \settowidth\dlf{$\displaystyle #1$}
            % The width of \dlf is the width of the lhs, with a displaystyle font
                \addtolength\dlf{\fboxsep+\fboxrule}
            % Add to it the distance to the box, and the width of the line of the box
                \hspace{-\dlf}
            % Move everything dlf units to the left, so that & #1 #2 is aligned under #1 & #2
                \boxed{#1 #2}
            % Put a box around lhs and rhs
        }
}

\title {Physikalisches Fortgeschrittenenpraktikum I\linebreak
Halbleiterbauelemente}
\author {\emph{Clemens Kurzenberg}\\212204196}
\date {16.04.2015}

\begin{document}

\maketitle

\begin{abstract}
Inhalt des Versuches ist die Analyse von Halbleiterdioden unter Einfluss von
Wechselstrom.
Dazu wird zunächst das Verhalten einer einzelnen Halbleiterdiode untersucht
und anschließend eine Gleichrichterschaltung.
\end{abstract}

\tableofcontents

\pagebreak
\listoffigures
\listoftables

\pagebreak
\section {Einführung}

Im Versuch untersucht werden eine einzelne Halbleiterdiode und eine
Gleichrichterschaltung.
Die verwendeten Geräte sind in Tabelle \ref{tab:devices} auf Seite
\pageref{tab:devices} verzeichnet.

\subsection {Halbleiterdiode}

Der erste Teilversuch erfolgt mit dem Aufbau in Abbildung \ref{fig:HL_Diode}.

\begin{figure}[!ht]
    \centering
    \begin{tikzpicture}[%show background rectangle,
    circuit ee IEC, circuit symbol lines/.style={draw,thick},
    font=\sffamily\upshape,
    >=latex % Voreinstellung für Pfeilspitzen
    ]
        \matrix (S) [
            matrix of nodes, nodes in empty cells,
            inner sep=0pt, outer sep=-.5\pgflinewidth,
            column sep=20mm, row sep = 7mm,
            nodes={minimum width=0pt}
        ]
        {
            &&&  \\
            &&&  \\
            &&&  \\
        };

        % Bauteile
        \draw (S-1-1)   to [resistor={info=$R_v$, name=R}](S-1-3)
                        to [diode={info=$D$, name=D}](S-3-3);

        % Leiter
        \draw   (S-1-3) -- (S-1-4)
                (S-3-1) -- (S-3-4);

        % Klemmen, Knoten
        \draw [fill=white]  (S-1-1) circle (2pt)
                            (S-1-4) circle (2pt)
                            (S-3-1) circle (2pt)
                            (S-3-4) circle (2pt);

        \draw [fill=black]  (S-1-3) circle (2pt)
                            (S-3-3) circle (2pt);

        \draw   (S-2-1) node {$U_e$}
                (S-2-4) node {$U_a$};
    \end{tikzpicture}
    \caption{Halbleiterdiode im Wechselstromkreis} \label{fig:HL_Diode}
\end{figure}

Das Ersatzschaltbild einer Halbleiterdiode ist in Abbildung \ref{fig:HL_Ersatz}
gezeigt.

\begin{figure}[!ht]
    \begin{subfigure}{\textwidth}
        \centering
        \begin{tikzpicture}[%show background rectangle,
        circuit ee IEC, circuit symbol lines/.style={draw,thick},
        font=\sffamily\upshape,
        >=latex % Voreinstellung für Pfeilspitzen
        ]
            \matrix (S) [
                matrix of nodes, nodes in empty cells,
                inner sep=0pt, outer sep=-.5\pgflinewidth,
                column sep=10mm, row sep = 9mm,
                nodes={minimum width=0pt}
            ]
            {
                &&&&&&&  \\
                &&&&&&&  \\
                &&&&&&&  \\
            };

            % Bauteile
            \draw (S-2-1)   to [inductor={info=$L_1$}] (S-2-3)
                            to [resistor={info=$R_B$, name=RB}](S-2-5)
                            to [resistor={info=$R_D$, name=RD}](S-2-7)
                  (S-1-5)   to [capacitor={info=$C_S$, name=CS}](S-1-7)
                  (S-3-5)   to [capacitor={info=$C_D$, name=CD}](S-3-7);

            % Leiter
            \draw   (S-2-7) -- (S-2-8)
                    (S-1-5) -- (S-3-5)
                    (S-1-7) -- (S-3-7);

            % Klemmen, Knoten
            \draw [fill=white]  (S-2-1) circle (2pt)
                                (S-2-8) circle (2pt);

            \draw [fill=black]  (S-2-5) circle (2pt)
                                (S-2-7) circle (2pt);
        \end{tikzpicture}
        \caption{Durchlassrichtung}
    \end{subfigure}
    \begin{subfigure}{\textwidth}
        \centering
        \begin{tikzpicture}[%show background rectangle,
        circuit ee IEC, circuit symbol lines/.style={draw,thick},
        font=\sffamily\upshape,
        >=latex % Voreinstellung für Pfeilspitzen
        ]
            \matrix (S) [
                matrix of nodes, nodes in empty cells,
                inner sep=0pt, outer sep=-.5\pgflinewidth,
                column sep=10mm, row sep = 9mm,
                nodes={minimum width=0pt}
            ]
            {
                &&&&&&&  \\
                &&&&&&&  \\
            };

            % Bauteile
            \draw (S-2-1)   to [inductor={info=$L_1$}] (S-2-3)
                            to [resistor={info=$R_B$, name=RB}](S-2-5)
                            to [resistor={info=$R_S$, name=RS}](S-2-7)
                  (S-1-5)   to [capacitor={info=$C_S$, name=CS}](S-1-7);

            % Leiter
            \draw   (S-2-7) -- (S-2-8)
                    (S-1-5) -- (S-2-5)
                    (S-1-7) -- (S-2-7);

            % Klemmen, Knoten
            \draw [fill=white]  (S-2-1) circle (2pt)
                                (S-2-8) circle (2pt);

            \draw [fill=black]  (S-2-5) circle (2pt)
                                (S-2-7) circle (2pt);
        \end{tikzpicture}
        \caption{Sperrrichtung}
    \end{subfigure}
    \caption[Ersatzschaltbild einer Halbleiterdiode]
    {Ersatzschaltbild einer Halbleiterdiode\\
    $L_1,R_B$ -- Induktivität und Widerstand des Bauelementes\\
    $C_S$ -- Sperrschichtkapazit\\
    $C_D$ -- Diffusionskapazität\\
    $R_D,R_S$ -- Durchlass- und Sperrwiderstand}
    \label{fig:HL_Ersatz}
\end{figure}

Dioden zeichnen sich durch ihre Eigenschaft aus, Strom nur in eine Richtung,
die Durchlassrichtung, zu leiten.
Bei angelegter Rechteckspannung stellt sich beim Umklappen in die Sperrrichtung
eine konstante Sperrkapazität ein, die der in etwa der angelegten Spannung
gleicht.
Dies geschieht nach einer endlichen Zeit, die benötigt wird,
um die Sperrschichtkapazität $C_S$ aufzuladen.
Diese Zeit $t_E$ heißt \emph{Sperrerholzeit}.
Nach dieser Zeit stellt sich zudem ein endlicher \emph{Dunkelstrom} durch
Diffusion von Ladungsträgern in Sperrrichtung ein.
Die \emph{Sperrverzögerungszeit} vom Umpolen der Spannung bis zum
Nulldurchgang $t_s$ wird im Experiment in Abhängigkeit des Durchlass- und
Sperrstroms ermittelt.
In dieser Zeit ist der Sperrstrom konstant und die überschüssigen Ladungen
der Sperrschichtkapazit werden abgetragen.

\subsection {Gleichrichterschaltung}

Die Gleichrichterschaltung wird nach Abbildung \ref{fig:Gleichrichter}
aufgebaut.

\begin{figure}[!ht]
    \centering
    \begin{tikzpicture}[%show background rectangle,
    circuit ee IEC, circuit symbol lines/.style={draw,thick},
    font=\sffamily\upshape,
    >=latex % Voreinstellung für Pfeilspitzen
    ]
        \matrix (S) [
            matrix of nodes, nodes in empty cells,
            inner sep=0pt, outer sep=-.5\pgflinewidth,
            column sep=10mm, row sep = 7mm,
            nodes={minimum width=0pt}
        ]
        {
            &&&&&&&&&  \\
            &&&&&&&&&  \\
            &&&&&&&&&  \\
            &&&&&&&&&  \\
            &&&&&&&&&  \\
        };

        \tikzset{
            Pfeil/.style={thick,shorten >=#1,shorten <=#1,->}, % für Peile
            UPfeil/.style={blue,Pfeil=#1,font={\sffamily\itshape}},% für Spannungspfeile
            IPfeil/.style={red,Pfeil=#1,font={\ttfamily\itshape}} % für Strompfeile
        }


        % Bauteile
        \draw (S-1-2)   to [diode={name=D1}](S-1-4) to (S-1-6)
                        to [resistor={info=$R_{39}$, name=R1}](S-1-8)
                        to [resistor={info=$R_{39'}$, name=R2}](S-3-8)
                        to [capacitor={info=$C_1$, name=C1}](S-5-8)
              (S-2-6)   to [capacitor={info=$C_2$, name=C2}](S-4-6)
              (S-2-9)   to [resistor={info=$R_L$, name=C2}](S-4-9)
              (S-2-4)   to [diode={name=D2}](S-2-2)
              (S-4-2)   to [diode={name=D3}](S-4-4)
              (S-5-4)   to [diode={name=D4}](S-5-2);

        % Leiter
        \draw   (S-1-8) -- (S-1-10)
                (S-5-4) -- (S-5-10)
                (S-1-9) -- (S-2-9)
                (S-4-9) -- (S-5-9)
                (S-1-6) -- (S-2-6)
                (S-4-6) -- (S-5-6)
                (S-1-2) -- (S-2-2)
                (S-4-2) -- (S-5-2)
                (S-2-1) -- (S-2-2)
                (S-4-1) -- (S-4-2)
                (S-5-4) -- (S-2-4)
                (S-4-4) -- (S-4-5) -- (S-1-5);

        % % Klemmen, Knoten
        \draw [fill=white]  (S-2-1) circle (2pt)
                            (S-1-10) circle (2pt)
                            (S-4-1) circle (2pt)
                            (S-5-10) circle (2pt);

        \draw [fill=black]  (S-1-9) circle (2pt)
                            (S-5-9) circle (2pt)
                            (S-1-8) circle (2pt)
                            (S-5-8) circle (2pt)
                            (S-1-6) circle (2pt)
                            (S-5-6) circle (2pt)
                            (S-2-2) circle (2pt)
                            (S-4-2) circle (2pt)
                            (S-5-4) circle (2pt)
                            (S-1-5) circle (2pt);

        \draw   (S-3-1) node {$U_e$}
                (S-3-10) node {$U_a$};

        \draw[UPfeil=0.25em] ([yshift=4mm]S-1-2.north) --
        ([yshift=4mm]S-1-3.north) node[midway, above]{$U^+$};
        \draw[UPfeil=0.25em] ([yshift=-4mm]S-5-4.north) --
        ([yshift=-4mm]S-5-3.north) node[midway, anchor=north]{$U^+$};
        \draw[UPfeil=0.25em] ([yshift=4mm]S-4-2.north) --
        ([yshift=4mm]S-4-3.north) node[midway, above]{$U^-$};
        \draw[UPfeil=0.25em] ([yshift=-4mm]S-2-4.north) --
        ([yshift=-4mm]S-2-3.north) node[midway, anchor=north]{$U^-$};
    \end{tikzpicture}
    \caption[Gleichrichterschaltung]
    {Gleichrichterschaltung\\
    Verlauf unterschiedlicher Stromrichtungen durch Pfeile mit $U^\pm$ angegeben}
    \label{fig:Gleichrichter}
\end{figure}

$U_e$ ist hierbei gegeben durch einen massenfreien Transformator.
$C_1$, $C_2$ und $R_{39'}$ sind im Verlauf des Versuches an einigen Stellen
entfernt, was durch $C=0$ oder $R=0$ angegeben wird.

Diese Schaltung soll dazu dienen, aus einem Wechselstromsignal ein möglichst
gutes Gleichstromsignal zu machen.
Die vier Dioden sorgen dafür, dass der Strom im restlichen Kreis nur in eine
Richtung fließen kann.

Wenn $C_2=0$ ist, dann ist $C_1$ ein sogenannter \emph{Ladekondensator},
der das Signal durch eine dem Strom entgegengesetzte Spannung erstmal abflacht.
Die entstehende Spannung beinhaltet immer noch Oszillationen und wird
\emph{Brummspannung} genannt.
In diesem Fall übernimmt $R_{39}$ nur die Rolle eines Messwiderstandes.

Ist $C_2\not=0$, übernimmt dieser die Rolle des Ladekondensators
und $C_1$ bildet mit $R_{39}$ einen Tiefpass, der das Signal weiter abflacht.
Dieser Tiefpass wird \emph{Siebglied} genannt.
as übrig bleiben sollte, ist eine gute Näherung an Gleichstrom.

Zur Untersuchung des Signals wird der \emph{Stromflusswinkel} betrachtet.
Er beschreibt das Verhältnis der Zeit $t_1$,
in der ein Signal übertragen wird gegenüber der gesamten Periodendauer $T$,
projiziert auf einen Vollwinkel.

\begin{equation}
    \Phi\,=\,\frac{t_1}{T}\,360^\circ
\end{equation}

\section {Durchführung}

\subsection {Halbleiterdiode}

Als Vorwiderstand wird $R_v=10~\mathrm{k\Omega}$ gewählt.
Der Strom wird durch einen Funktionsgenerator als Rechteckspannung der
Frequenz $\nu=2.194\mathrm{kHz}$ geliefert.

Im Folgenden wird mit $U_e$ die eingehende Spannung, mit $U_{p-p}$ deren
Peak-to-Peak-Wert, mit $U_o$ der Betrag über $0~\mathrm V$ und $U_u$ der
Betrag unter $0~\mathrm V$ bezeichnet.
$U_a$ ist die ausgehende Spannung, $U_R$ die, die am Widerstand abfällt,
$U_D$ die Durchlassspannung an der Diode und $U_S=U_u$ deren Sperrspannung.
Analog sind $I_D$ und $I_S$ Durchlass- und Sperrstrom der Diode.

Gemessen werden jeweils $U_{p-p}$, $U_D$ und $t_s$.
Zunächst wird $U_o=1.51~\mathrm V$ konstant gehalten und $U_{p-p}$ variiert.
Dadurch ändert sich $U_u=U_S$ und für den Strom gilt
$I_S=\frac{U_S}{R_v}=\frac{U_{p-2}-U_o}{R_v}$.
Anschließend wird $U_u=1.49~\mathrm V$ konstantgehalten, $U_{p-p}$ und
analog $I_D=\frac{U_o-U_D}{R_v}=\frac{U_{p-2}-U_u-U_D}{R_v}$ berechnet.

Die Messung ist in Tabelle \ref{tab:Diode_Uo} und \ref{tab:Diode_Uu}
protokolliert und Ergebnisse sind in Abbildung \ref{fig:HL_t_s} dargestellt.

\begin{table}[!ht]
    \centering
    \caption{Messwerte $t_s(I_D)$ für $U_o=1.51~\mathrm V=const$}
    \label{tab:Diode_Uo}
    \begin{tabular}{l|l|l|l}
        $U_{p-p}/\mathrm V$&$t_s/\mathrm{\mu s}$
        &$U_D/\mathrm{\mu s}$&$I_D/\mathrm{mI}$\\
        \hline
        7.1&1.09&430&0.56\\
        6.6&1.374&440&0.51\\
        6&1.516&440&0.45\\
        5.5&1.788&425&0.4\\
        5.1&2.25&425&0.36\\
        4.48&2.55&420&0.298\\
        3.99&3.245&430&0.249\\
        3.51&3.91&430&0.201\\
        3.06&5.3&420&0.156\\
        2.49&7.31&420&0.099
    \end{tabular}
    \medskip
    \caption{Messwerte $t_s(I_S)$ für $U_u=1.49~\mathrm V=const$}
    \label{tab:Diode_Uu}
    \begin{tabular}{l|l|l|l}
        $U_{p-p}/\mathrm V$&$t_s/\mathrm{\mu s}$
        &$U_D/\mathrm{\mu s}$&$I_S/\mathrm{mI}$\\
        \hline
        7.04&15.98&490&0.505\\
        6.51&14.64&490&0.452\\
        6.03&13.56&474&0.4056\\
        5.51&12&461&0.3549\\
        5.03&11.42&455&0.3075\\
        4.5&9.84&448&0.2552\\
        4.02&8.08&445&0.2075\\
        3.5&6.02&416&0.1584\\
        3.02&3.85&406&0.1114\\
        2.49&1.746&368&0.0622\\
    \end{tabular}
\end{table}

\begin{figure}[!ht]
    \begin{subfigure}{\textwidth}
        \centering
        \includegraphics[width=0.7\textwidth]{graphics/diode_I_t_durchlass.eps}
        \caption{$t_s(I_S)$ bei $I_D=\text{const}$}
    \end{subfigure}
    \begin{subfigure}{\textwidth}
        \centering
        \includegraphics[width=0.7\textwidth]{graphics/diode_I_t_sperr.eps}
        \caption{$t_s(I_D)$ bei $I_S=\text{const}$}
    \end{subfigure}
    \caption{Verhalten von $t_s$ in Abhängigkeit von Sperrstrom $I_S$ und
    Duchlassstrom $I_D$}
    \label{fig:HL_t_s}
\end{figure}

Die Daten zum konstanten Durchlassstrom erinnern an exponentielles Verhalten.
Daher wird ein Fit der Form $t_{s,fit}(I_S)=a_S\,\exp\left(b_S\,I_S\right)$
mit folgenden Ergebnissen durchgeführt:

\begin{align}
    a_S \,&=\, \left(11.1\pm0.7\right)~\mathrm{\mu s}\\
          &=\, 11.1\,\left(1\pm5.5\%\right)~\mathrm{\mu s}\\
    b_S \,&=\, \left(-4.7\pm0.3\right)~\mathrm{mA^{-1}}\\
          &=\, -4.7\,\left(1\pm5.8\%\right)~\mathrm{mA^{-1}}
\end{align}

Bei konstantem Sperrstrom zeigt sich annähernd lineares Verhalten mit
$t_{s,fit}(I_D)=a_D+b_D\,I_D$.

\begin{align}
    a_D \,&=\, \left(0.9\pm0.6\right)~\mathrm{\mu s}\\
          &=\, 0.9\,\left(1\pm61.4\%\right)~\mathrm{\mu s}\\
    b_D \,&=\, \left(31.4\pm1.7\right)~\mathrm{ms\,A^{-1}}\\
          &=\, 31.4\,\left(1\pm5.3\%\right)~\mathrm{ms\,A^{-1}}
\end{align}

\subsection {Gleichrichterschaltung}

Für die massenfreien Spannungsversorgung wird ein Transformator von Netzspannung
$220~\mathrm V/50~\mathrm{Hz}$ auf $6~\mathrm V/50~\mathrm{Hz}$ verwendet.
Der Lastwiderstand beträgt $R_L=10~\mathrm{k\Omega}$.

Zunächst sei $C_2=0$ und $R_{39}=R_{39}'=39~\Omega$.
Gemessen werden die Spannungen über die Messwiderstände $R_{39}$, $R_{39}'$
und den Lastwiderstand $R_L$,
namentlich $U_R$, $U_C$ und $U_a$.
Das Signal $U_C$ und $U_a$ werden invertiert,
da man an diesen Stellen gegen den Strom misst.
Aus diesem Messgrößen lassen sich die Ströme über Dioden, Kondensator und
Signalausgang berechnen.

\begin{align}
    I_D \,&=\,\frac{U_R}{R_{39}}\\
    I_C \,&=\,\frac{U_C}{R_{39}'}\\
    I_a \,&=\,\frac{U_a}{R_L}
\end{align}

Für $C_1=0$ und $C_1=10~\mathrm{\mu F}$ sind die Spannungssignale in Abbildung
\ref{fig:Gleichrichter_U} und die Ströme in Abbildung
\ref{fig:Gleichrichter_I} ab Seite \pageref{fig:Gleichrichter_U} dargestellt.

Für $C_1=0$ ist das Signal an Widerstand $C_{39}$ und Ausgang zu gleichen Zeiten
der Länge $t_{0,0}=608~\mathrm{ns}$ null.
Aufgrund der angelegten $50~\mathrm{Hz}$ und halbperiodischer Signalumpolung
ist $T=10~\mathrm{ms}$.
Damit ist der Stromwinkel hier $\Phi_0=338^\circ$.

Für $C_1=10~\mathrm{\mu F}$ ist $U_a$ nie null,
sodass $\Phi_{10~\mathrm{\mu F},a}=360^\circ$ ist.
Am Widerstand ist das Signal jedoch für eine Dauer
$t_{10~\mathrm{\mu F},0}=4.84~\mathrm{ms}$ null.
An den Dioden ist der Stromwinkel also
$\Phi_{10~\mathrm{\mu F},D}=185.76^\circ$.

Um zu sehen, wie gut der Gleichrichter funktioniert,
wird die Ausgangsspannung $U_a$ für verschiedene Fälle aufgenommen,
in denen $R_{39}'=0$ ist.
Der Vergleich ist in Tabelle \ref{tb:Gleichrichter} auf Seite
\pageref{tb:Gleichrichter} protokolliert.
Die Spannungsverläufe sind in Abbildung \ref{fig:Gleichrichter_U2} aufgetragen.

\begin{table}[!ht]
    \centering
    \caption{Vergleich der Gleichrichterschaltung für verschiedene Parameter
    mit $R_{39}'=0$}
    \label{tb:Gleichrichter}
    \begin{tabular}{r|r|l|l|l}
        $C_1/\mathrm{\mu F}$&$C_2/\mathrm{\mu F}$&$U_{p-p}/\mathrm V$
        &$U_{avg}/\mathrm V$&$U_{top}/\mathrm V$\\
        \hline
        0&0&8.8&5.15&8.8\\
        10&0&3.9&6.68&8.3\\
        100&0&0.70&6.51&7.7\\
        100&100&0.23&8.06&8.15\\
    \end{tabular}
\end{table}


\pagebreak
\section {Auswertung}

\subsection {Halbleiterdiode}

Für das Verhalten der Sperrverzögerungszeit in Abhängigkeit vom
Durchlassstrom $t_s(I_D)$ wurde innerhalb des betrachteten Messbereiches
$I_D\in(0.07,0.5)$ näherungsweise lineares Verhalten festgestellt.

Für das Verhalten in Abhängigkeit vom Sperrstrom $I_s(I_S)$ hingegen
sieht es im Messbereich $I_S\in(0.1,0.55)$ näherungsweise nach exponentiellem
Verhalten aus.

\begin{empheq}[box=\fbox]{align}
    t_s(I_D)\,&\propto\,I_D\\
    \ln t_s(I_S)\,&\propto\,-I_S
\end{empheq}

Dies ist in sofern logisch, dass die Sperrschichtkapazität $C_S$ in der Diode
von der angelegten Spannung abhängig ist,
mit höherer Spannung in Flussrichtung nimmt die Kapazität zu,
während sie in Sperrrichtung näherungsweise konstant ist.
Somit sorgt eine größere Durchlassspannung $U_D\propto I_D$ für eine höhere
Sperrschichtkapazität $C_S$, in der mehr Ladungsträger gespeichert werden,
die beim Umpolen wieder abgetragen werden müssen.
Das Abtragen von mehr Ladungsträgern benötigt eine längere Zeit und somit
steigt die Zeit $t_s(I_D)$ mit wachsendem $U_D\propto I_D$.
Das lineare Verhalten ist lediglich eine Approximation und man kann im 
Allgemeinen einen komplexeren theoretischen Verlauf vermuten,
zumal die Diffusionskapazit $C_D$ ebenfalls eine signifikante Rolle spielen
kann.

Eine größere Sperrspannung $U_S\propto I_S$ wirkt sich additiv auf die
Spannung innerhalb der Sperrschichtkapazität aus,
sodass die Ladungsträger schneller abgetragen werden können.
Ein exponentielles Verhalten erscheint in sofern realistisch,
dass selbst bei sehr hohen Sperrspannungen $U_S$ noch eine endliche Zeit,
die größer als Null ist, benötigt wird,
um alle Ladungsträger abzutragen und die Sperrschichtkapazität durch
ihr inneres Feld immer noch selbstständig Ladungsträger innerhalb einer
endlichen Zeit abträgt, wenn keine äußere Spannung $U_S$ anliegt.
Letzteres ist hierbei lediglich eine Näherung,
da die Entladefunktion eines Kondensators nie Null erreicht,
jedoch ist diese Approximation hier ausreichend.

\subsection {Gleichrichterschaltung}

Die Messergebnisse verdeutlichen gut, wie die Gleichrichterschaltung
funktioniert.

In Abbildung \ref{fig:Gleichrichter_I_0F} ist der Stromverlauf für
$C_1=C_2=0$ darfestellt.
Die angelegte Frequenz ist ein Sinus mit $\nu=50~\mathrm{Hz}$,
also einer Periodendauer von $20~\mathrm{ms}$.
Der Gleichrichter erlaubt nur eine Stromrichtung,
indem der gegenpolige Strom durch Dioden so geleitet wird,
dass er im restlichen Kreis in die gleiche Richtung fließt
wie geichpoliger Strom.
Dies ist in Abbildung \ref{fig:Gleichrichter} eingezeichnet.
und theoretisch äquivalent zu einer Sinusbetragsfunktion.
Die Periodendauer wird also halbiert auf:

\begin{equation}
    \boxed{T\,=\,10\mathrm{ms}}
\end{equation}

Der ermittelte Stromflusswinkel für diesen Fall gleicht jedoch nicht einem
Vollwinkel, sondern wurde lediglich ermittelt als:

\begin{equation}
    \boxed{\Phi_0\,=\,338^\circ}
\end{equation}

Das liegt daran, dass an einer Diode erst eine bestimmte Schleusenspannung
$U_{sch}$
in Flussrichtung überschritten werden muss, damit diese Strom leitet.
Solange der Betrag der eingehenden Spannung $U_e$ also unter diesem Wert liegt,
fließt im Gleichrichter kein Strom.
Die Schleusenspannung lässt sich über den Stromflusswinkel ausrechnen.
$360^\circ-\Phi_0$ entspricht dem Anteil der Periodendauer,
in der kein Strom durch die Diode fließt.
Da die Spannung in der Zeit gleichermaßen auf 0 absinkt und wieder zu
$U_{sch}$, muss dieser halbiert werden und da es sich nur um einen
halben Sinus handelt, wird er erneut halbiert.
Es liegen hinter dem Transformator $U_{e,eff}=6~\mathrm V$ an.

\begin{align}
    U_{sch}\,&=\,\sqrt2\,U_{e,eff}\,\sin\frac{360^\circ-\Phi_0}{4}\\
    \alignedbox{U_{sch}\,}{=\,0.81~\mathrm V}
\end{align}

Die typischen Werte für Siliziumdioden liegen mit
$U_{sch,Si}\approx0.7~\mathrm V$ recht nah.

Wird nun ein Ladekondensator mit $C_1=10~\mathrm{\mu F}$ eingebaut,
ändert sich die Situation, wie in Abbildung \ref{fig:Gleichrichter_I_10uF}
zu sehen ist.
Im Zeitraum $t\in(0,\frac{T}{2}=5~\mathrm{ms}$) innerhalb einer Periode
lädt sich der Kondensator auf.
Dies ist die Zeit, in der die Spannung $U_e$ monoton ansteigt.
Dies geschieht durch den Strom, der durch die Dioden fließt.
Für $t\in(\frac{T}{2}=5~\mathrm{ms},T=10~\mathrm{ms})$ sinkt die Spannung $U_e$,
sodass das Feld im Kondensator überwiegt.
Dieser entlädt sich und führt zu einem dauerhaft positiven Strom $I_a\not=0$.
Während der Kondensator sich entlädt,
müsste der Strom im Schaltbild links vom Kondensator entgegen der Spannung
$U_e$ fließen,
was jedoch in Sperrrichtung der Dioden wäre, sodass $I_D$ in diser Zeit null
ist.
Das Ergebnis ist ein voller Stromflusswinkel über $I_a$ und ungefähr ein
halber Stromflusswinkel über $I_D$.

\begin{empheq}[box=\fbox]{align}
    \Phi_{10~\mu F,a}\,&=\,360^\circ\\
    \Phi_{10~\mu F,D}\,&=\,186^\circ
\end{empheq}

Abbildung \ref{fig:Gleichrichter_U2} stellt die unterschiedlichen
Spannungsverläufe von $U_a\propto I_a$ dar
und zeigt gut, dass die Spannung im zweiten Fall nie null wird.
Dennoch schwankt die Spannung noch stark und ist weit von Gleichstrom entfernt.
Eine höhere Kapazität $C_1$ glättet das Signal etwas,
doch erst ein Tiefpassfilter, bzw. Siebglied, glättet das Signal so,
dass ein Signal geringer Varianz entsteht.
Tabelle \ref{tb:Gleichrichter} zeigt, wie die Signalbreite $U_{p-p}$
sich in den unterschiedlichen Fällen verändert.
Mit Ladekondensator und Siebglied wird eine Signalbreite von lediglich
$U_{p-p}=0.23~\mathrm V$ erreicht.
Außerdem erreicht man mit Siebgleich einen hohen Spannungswert von
$U_{avg}=8.06~\mathrm V$,
während er durch einfaches Erhöhen der Kapazität am Ladekondensator absinkt.
Man erreicht so also einen geringeren Leistungsverlust.
Der Spitzenwert von $U_e$ liegt bei
$U_{e,amp}=\sqrt2\cdot6~\mathrm V=8.49~\mathrm V$.
Es gibt mit Lastwiderstand, Siebglied und Ladekondensator also nur etwa ein
halbes Volt Verlust gegenüber diesem Wert.


\pagebreak
% \section {Quellen}
% \begin{thebibliography}{999}
% \bibitem {WalterHerms} G. Walter und G. Herms, Einführung in die Behandlung von Messfehlern -- Ein Leitfaden für das Praktikum der Physik, Universität Rostock 2006d
% \end{thebibliography}

\section {Anhang}

\begin{table}[ht!]
    \centering
    \caption{verwendete Geräte} \label{tab:devices}
    \begin{tabular}{lr}
        Arbeitsplatz & 9\\
        Oszilloskop&Agilent Technologies Infiicision MSO-X 2002A\\
        Funktionsgenerator&FG1617 Function Generator\\
        Mini $\Omega$ Dekade&RT 1-1000 SAB\\
        Steckbrett & 4
    \end{tabular}
\end{table}

\begin{figure}[!ht]
    \begin{subfigure}{\textwidth}
        \centering
        \includegraphics[width=0.7\textwidth]{graphics/gleichrichter_0F.eps}
        \caption{$C_1=0$}
    \end{subfigure}
    \begin{subfigure}{\textwidth}
        \centering
        \includegraphics[width=0.7\textwidth]{graphics/gleichrichter_10uF.eps}
        \caption{$C_1=10~\mathrm{\mu F}$}
    \end{subfigure}
    \caption{Spannungen im Gleichrichter mit $C_2=0$ und $R_{39}'=R_{39}$ für
        die Fälle $C_1=0$ und $C_1=10~\mathrm{\mu F}$}
    \label{fig:Gleichrichter_U}
\end{figure}

\begin{figure}[!ht]
    \begin{subfigure}{\textwidth}
        \centering
        \includegraphics[width=0.7\textwidth]
            {graphics/gleichrichter_strom_0F.eps}
        \caption{$C_1=0$}
        \label{fig:Gleichrichter_I_0F}
    \end{subfigure}
    \begin{subfigure}{\textwidth}
        \centering
        \includegraphics[width=0.7\textwidth]
            {graphics/gleichrichter_strom_10uF.eps}
        \caption{$C_1=10~\mathrm{\mu F}$}
        \label{fig:Gleichrichter_I_10uF}
    \end{subfigure}
    \caption{Ströme im Gleichrichter mit $C_2=0$ und $R_{39}'=R_{39}$ für
        die Fälle $C_1=0$ und $C_1=10~\mathrm{\mu F}$}
    \label{fig:Gleichrichter_I}
\end{figure}

\begin{figure}
    \centering
    \includegraphics[width=\textwidth]{graphics/gleichrichter_spannungen.eps}
    \caption{Spannungsverläufe $U_a$ im Gleichrichter für unterschiedliche
    $C_1$ und $C_2$}
    \label{fig:Gleichrichter_U2}
\end{figure}


\end{document}
%sagemathcloud={"zoom_width":100}
